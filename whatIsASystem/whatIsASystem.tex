\documentclass{ximera}
\title{What Is A System}
\begin{document}
\begin{abstract}
Another Ximera activity
\end{abstract}
\maketitle

To introduce the concept of a system of ODEs, let's consider the \emph{Lotka-Volterra predator-prey model}, which describes two animal species in the wild, traditionally referred to as ``rabbits'' (the prey) and ``foxes'' (the predator). Denoting by $R(t)$ and $F(t)$ the rabbit and fox populations, respectively, we use the following differential equations to model the dynamics of the two species in the wild:
\begin{align}
\label{rabbit-eq}R'&=aR-bRF\\
\label{fox-eq}F'&=cF+dRF
\end{align}
How do ecologists come up with this model? This is not a simple question, since there usually are many possible ways of modeling a real-world system. As a rule of thumb, scientists try to use the simplest possible model as a first approximation. For the predator prey model, the following are reasonable hypothesis:

\end{document}
