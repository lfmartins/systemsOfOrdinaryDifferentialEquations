\documentclass{ximera}
\title{What Is A System}
\begin{document}
\begin{abstract}
Another Ximera activity
\end{abstract}
\maketitle

To introduce the concept of a system of ODEs, let's consider the \emph{Lotka-Volterra predator-prey model}, which describes two animal species in the wild, traditionally referred to as ``rabbits'' (the prey) and ``foxes'' (the predator). Denoting by $R(t)$ and $F(t)$ the rabbit and fox populations, respectively, we use the following differential equations to model the dynamics of the two species in the wild:
\begin{align}
\label{rabbit-eq}R'&=aR-bRF\\
\label{fox-eq}F'&=cF+dRF
\end{align}
How do ecologists come up with this model? This is not a simple question, since there usually are many possible ways of modeling a real-world system. As a rule of thumb, scientists try to use the simplest possible model as a first approximation. For the predator prey model, the following are reasonable hypothesis:

%\begin{itemize}
%\item 
In the absence of predators, the prey population grows without limits. We can model this by an exponential growth model:
\[
R'=aR\text{ where $a>0$}.
\]
%\item 
In the absence of prey, the predator population decays to zero, since they don't have any available food. This can be modeled by an exponential decay model:
\[
F'=-cF\text{ where $b>0$.}
\]
%\item
To model the interactions between the two populations, we assume that the number of possible ``predation situations'' is proportional to the number of possible ``encounters'' between individuals of the two populations, which can be approximated by the product $RF$.
%\item 
Since predation is disadvantageous to the prey population, we subtract a term proportional to $RF$ to the prey equation:
\[
R'=aR-bRF\text{ where $a,c>0$}.
\]
%\item 
Since predation is beneficial to the predator population, we add a term proportional to $RF$ to the predator population:
\[
F'=-cF+dRF\text{ where $b,d>0$.}
\]
%\end{itemize}
Putting all together, we get the pair of equations~\ref{rabbit-eq},~\ref{fox-eq}.

\begin{problem}

Let's consider now consider two populations $x$ and $y$ that are in competition with each other. We make the following assumptions:
\begin{itemize}
\item In isolation, each of the populations grows exponentially. We denote the growth rates for $x$ and $y$ by $a>0$ and $c>0$, respectively. 
\item The effect of competition to each population is proportional to the product of the populations sizes, $xy$. We denote the corresponding proportionality constants for populations $x$ and $y$ by $b>0$ and $d>0$, respectively.
\end{itemize}

\begin{question}
$x'= \answer{ax-bxy}$
\end{question}
\begin{question}
$y'= \answer{cy-dxy}$
\end{question}
\end{problem}

\end{document}
