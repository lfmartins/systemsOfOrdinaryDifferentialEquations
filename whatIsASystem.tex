% !TEX root = ../systemsOfOrdinaryDifferentialEquations.tex
\documentclass{ximera}


%\usepackage{todonotes}

\newcommand{\todo}{}

\usepackage{esint} % for \oiint
\ifxake%%https://math.meta.stackexchange.com/questions/9973/how-do-you-render-a-closed-surface-double-integral
\renewcommand{\oiint}{{\large\bigcirc}\kern-1.56em\iint}
\fi


\graphicspath{
  {./}
  {ximeraTutorial/}
  {basicPhilosophy/}
  {functionsOfSeveralVariables/}
  {normalVectors/}
  {lagrangeMultipliers/}
  {vectorFields/}
  {greensTheorem/}
  {shapeOfThingsToCome/}
  {dotProducts/}
  {../productAndQuotientRules/exercises/}
  {../normalVectors/exercisesParametricPlots/}
  {../continuityOfFunctionsOfSeveralVariables/exercises/}
  {../partialDerivatives/exercises/}
  {../chainRuleForFunctionsOfSeveralVariables/exercises/}
  {../commonCoordinates/exercisesCylindricalCoordinates/}
  {../commonCoordinates/exercisesSphericalCoordinates/}
  {../greensTheorem/exercisesCurlAndLineIntegrals/}
  {../greensTheorem/exercisesDivergenceAndLineIntegrals/}
  {../shapeOfThingsToCome/exercisesDivergenceTheorem/}
  {../greensTheorem/}
  {../shapeOfThingsToCome/}
}

\newcommand{\mooculus}{\textsf{\textbf{MOOC}\textnormal{\textsf{ULUS}}}}

\usepackage{tkz-euclide}\usepackage{tikz}
\usepackage{tikz-cd}
\usetikzlibrary{arrows}
\tikzset{>=stealth,commutative diagrams/.cd,
  arrow style=tikz,diagrams={>=stealth}} %% cool arrow head
\tikzset{shorten <>/.style={ shorten >=#1, shorten <=#1 } } %% allows shorter vectors

\usetikzlibrary{backgrounds} %% for boxes around graphs
\usetikzlibrary{shapes,positioning}  %% Clouds and stars
\usetikzlibrary{matrix} %% for matrix
\usepgfplotslibrary{polar} %% for polar plots
\usepgfplotslibrary{fillbetween} %% to shade area between curves in TikZ
\usetkzobj{all}
\usepackage[makeroom]{cancel} %% for strike outs
%\usepackage{mathtools} %% for pretty underbrace % Breaks Ximera
%\usepackage{multicol}
\usepackage{pgffor} %% required for integral for loops



%% http://tex.stackexchange.com/questions/66490/drawing-a-tikz-arc-specifying-the-center
%% Draws beach ball
\tikzset{pics/carc/.style args={#1:#2:#3}{code={\draw[pic actions] (#1:#3) arc(#1:#2:#3);}}}



\usepackage{array}
\setlength{\extrarowheight}{+.1cm}
\newdimen\digitwidth
\settowidth\digitwidth{9}
\def\divrule#1#2{
\noalign{\moveright#1\digitwidth
\vbox{\hrule width#2\digitwidth}}}





\newcommand{\RR}{\mathbb R}
\newcommand{\R}{\mathbb R}
\newcommand{\N}{\mathbb N}
\newcommand{\Z}{\mathbb Z}

\newcommand{\sagemath}{\textsf{SageMath}}


%\renewcommand{\d}{\,d\!}
\renewcommand{\d}{\mathop{}\!d}
\newcommand{\dd}[2][]{\frac{\d #1}{\d #2}}
\newcommand{\pp}[2][]{\frac{\partial #1}{\partial #2}}
\renewcommand{\l}{\ell}
\newcommand{\ddx}{\frac{d}{\d x}}

\newcommand{\zeroOverZero}{\ensuremath{\boldsymbol{\tfrac{0}{0}}}}
\newcommand{\inftyOverInfty}{\ensuremath{\boldsymbol{\tfrac{\infty}{\infty}}}}
\newcommand{\zeroOverInfty}{\ensuremath{\boldsymbol{\tfrac{0}{\infty}}}}
\newcommand{\zeroTimesInfty}{\ensuremath{\small\boldsymbol{0\cdot \infty}}}
\newcommand{\inftyMinusInfty}{\ensuremath{\small\boldsymbol{\infty - \infty}}}
\newcommand{\oneToInfty}{\ensuremath{\boldsymbol{1^\infty}}}
\newcommand{\zeroToZero}{\ensuremath{\boldsymbol{0^0}}}
\newcommand{\inftyToZero}{\ensuremath{\boldsymbol{\infty^0}}}



\newcommand{\numOverZero}{\ensuremath{\boldsymbol{\tfrac{\#}{0}}}}
\newcommand{\dfn}{\textbf}
%\newcommand{\unit}{\,\mathrm}
\newcommand{\unit}{\mathop{}\!\mathrm}
\newcommand{\eval}[1]{\bigg[ #1 \bigg]}
\newcommand{\seq}[1]{\left( #1 \right)}
\renewcommand{\epsilon}{\varepsilon}
\renewcommand{\phi}{\varphi}


\renewcommand{\iff}{\Leftrightarrow}

\DeclareMathOperator{\arccot}{arccot}
\DeclareMathOperator{\arcsec}{arcsec}
\DeclareMathOperator{\arccsc}{arccsc}
\DeclareMathOperator{\si}{Si}
\DeclareMathOperator{\scal}{scal}
\DeclareMathOperator{\sign}{sign}


%% \newcommand{\tightoverset}[2]{% for arrow vec
%%   \mathop{#2}\limits^{\vbox to -.5ex{\kern-0.75ex\hbox{$#1$}\vss}}}
\newcommand{\arrowvec}[1]{{\overset{\rightharpoonup}{#1}}}
%\renewcommand{\vec}[1]{\arrowvec{\mathbf{#1}}}
\renewcommand{\vec}[1]{{\overset{\boldsymbol{\rightharpoonup}}{\mathbf{#1}}}}
\DeclareMathOperator{\proj}{\vec{proj}}
\newcommand{\veci}{{\boldsymbol{\hat{\imath}}}}
\newcommand{\vecj}{{\boldsymbol{\hat{\jmath}}}}
\newcommand{\veck}{{\boldsymbol{\hat{k}}}}
\newcommand{\vecl}{\vec{\boldsymbol{\l}}}
\newcommand{\uvec}[1]{\mathbf{\hat{#1}}}
\newcommand{\utan}{\mathbf{\hat{t}}}
\newcommand{\unormal}{\mathbf{\hat{n}}}
\newcommand{\ubinormal}{\mathbf{\hat{b}}}

\newcommand{\dotp}{\bullet}
\newcommand{\cross}{\boldsymbol\times}
\newcommand{\grad}{\boldsymbol\nabla}
\newcommand{\divergence}{\grad\dotp}
\newcommand{\curl}{\grad\cross}
%\DeclareMathOperator{\divergence}{divergence}
%\DeclareMathOperator{\curl}[1]{\grad\cross #1}
\newcommand{\lto}{\mathop{\longrightarrow\,}\limits}

\renewcommand{\bar}{\overline}

\colorlet{textColor}{black}
\colorlet{background}{white}
\colorlet{penColor}{blue!50!black} % Color of a curve in a plot
\colorlet{penColor2}{red!50!black}% Color of a curve in a plot
\colorlet{penColor3}{red!50!blue} % Color of a curve in a plot
\colorlet{penColor4}{green!50!black} % Color of a curve in a plot
\colorlet{penColor5}{orange!80!black} % Color of a curve in a plot
\colorlet{penColor6}{yellow!70!black} % Color of a curve in a plot
\colorlet{fill1}{penColor!20} % Color of fill in a plot
\colorlet{fill2}{penColor2!20} % Color of fill in a plot
\colorlet{fillp}{fill1} % Color of positive area
\colorlet{filln}{penColor2!20} % Color of negative area
\colorlet{fill3}{penColor3!20} % Fill
\colorlet{fill4}{penColor4!20} % Fill
\colorlet{fill5}{penColor5!20} % Fill
\colorlet{gridColor}{gray!50} % Color of grid in a plot

\newcommand{\surfaceColor}{violet}
\newcommand{\surfaceColorTwo}{redyellow}
\newcommand{\sliceColor}{greenyellow}




\pgfmathdeclarefunction{gauss}{2}{% gives gaussian
  \pgfmathparse{1/(#2*sqrt(2*pi))*exp(-((x-#1)^2)/(2*#2^2))}%
}

%%%% Added by Felipe
%\usepackage{pgfplots}
%\usepgfplotslibrary{external}
%\tikzexternalize



%%%%%%%%%%%%%
%% Vectors
%%%%%%%%%%%%%

%% Simple horiz vectors
\renewcommand{\vector}[1]{\left\langle #1\right\rangle}


%% %% Complex Horiz Vectors with angle brackets
%% \makeatletter
%% \renewcommand{\vector}[2][ , ]{\left\langle%
%%   \def\nextitem{\def\nextitem{#1}}%
%%   \@for \el:=#2\do{\nextitem\el}\right\rangle%
%% }
%% \makeatother

%% %% Vertical Vectors
%% \def\vector#1{\begin{bmatrix}\vecListA#1,,\end{bmatrix}}
%% \def\vecListA#1,{\if,#1,\else #1\cr \expandafter \vecListA \fi}

%%%%%%%%%%%%%
%% End of vectors
%%%%%%%%%%%%%

%\newcommand{\fullwidth}{}
%\newcommand{\normalwidth}{}



%% makes a snazzy t-chart for evaluating functions
%\newenvironment{tchart}{\rowcolors{2}{}{background!90!textColor}\array}{\endarray}

%%This is to help with formatting on future title pages.
\newenvironment{sectionOutcomes}{}{}



%% Flowchart stuff
%\tikzstyle{startstop} = [rectangle, rounded corners, minimum width=3cm, minimum height=1cm,text centered, draw=black]
%\tikzstyle{question} = [rectangle, minimum width=3cm, minimum height=1cm, text centered, draw=black]
%\tikzstyle{decision} = [trapezium, trapezium left angle=70, trapezium right angle=110, minimum width=3cm, minimum height=1cm, text centered, draw=black]
%\tikzstyle{question} = [rectangle, rounded corners, minimum width=3cm, minimum height=1cm,text centered, draw=black]
%\tikzstyle{process} = [rectangle, minimum width=3cm, minimum height=1cm, text centered, draw=black]
%\tikzstyle{decision} = [trapezium, trapezium left angle=70, trapezium right angle=110, minimum width=3cm, minimum height=1cm, text centered, draw=black]



\title{What is a System}

\begin{document}

\begin{abstract}
This activity describes what is a system of ordinary differential equations.
\end{abstract}

\maketitle

\begin{sectionOutcomes}
After completing this section, students should be able to do the following:

\begin{itemize}
\item Recognize a system of ODEs.
\item Interpret a system of ODEs as a vector field.
\item Understand the representation of a vector field in the phase plane.
\end{itemize}

\end{sectionOutcomes}

To introduce the concept of a system of ODEs, let's consider the \emph{Lotka-Volterra predator-prey model}, which describes two animal species in the wild, traditionally referred to as ``rabbits'' (the prey) and ``foxes'' (the predator). Denoting by $R(t)$ and $F(t)$ the rabbit and fox populations, respectively, we use the following differential equations to model the dynamics of the two species in the wild:
\begin{align}
\label{rabbit-eq}R'&=aR-bRF\\
\label{fox-eq}F'&=cF+dRF
\end{align}
How do ecologists come up with this model? This is not a simple question, since there usually are many possible ways of modeling a real-world system. As a rule of thumb, scientists try to use the simplest possible model as a first approximation. For the predator prey model, the following are reasonable hypothesis:

%\begin{itemize}
%\item 
In the absence of predators, the prey population grows without limits. We can model this by an exponential growth model:
\[
R'=aR\text{ where $a>0$}.
\]
%\item 
In the absence of prey, the predator population decays to zero, since they don't have any available food. This can be modeled by an exponential decay model:
\[
F'=-cF\text{ where $b>0$.}
\]
%\item
To model the interactions between the two populations, we assume that the number of possible ``predation situations'' is proportional to the number of possible ``encounters'' between individuals of the two populations, which can be approximated by the product $RF$.
%\item 
Since predation is disadvantageous to the prey population, we subtract a term proportional to $RF$ to the prey equation:
\[
R'=aR-bRF\text{ where $a,c>0$}.
\]
%\item 
Since predation is beneficial to the predator population, we add a term proportional to $RF$ to the predator population:
\[
F'=-cF+dRF\text{ where $b,d>0$.}
\]
%\end{itemize}
Putting all together, we get the pair of equations~\ref{rabbit-eq},~\ref{fox-eq}.

\begin{problem}

Let's consider now consider two populations $x$ and $y$ that are in competition with each other. We make the following assumptions:
\begin{itemize}
\item In isolation, each of the populations grows exponentially. We denote the growth rates for $x$ and $y$ by $a>0$ and $c>0$, respectively. 
\item The effect of competition to each population is proportional to the product of the populations sizes, $xy$. We denote the corresponding proportionality constants for populations $x$ and $y$ by $b>0$ and $d>0$, respectively.
\end{itemize}

\begin{question}
$x'= \answer{ax-bxy}$
\end{question}
\begin{question}
$y'= \answer{cy-dxy}$
\end{question}
\end{problem}

%\begin{explanation} 
%In the absence of competition, the dynamics for each population is given by an exponential growth model:
%\begin{align*}
%x'&=ax\\
%y'&=cy.
%\end{align*}
%Competition has a negative effect on the growth of each population, so we need to subtract a term proportional to $xy$ to each equation, obtaining:
%\begin{align*}
%x'&=ax-bxy\\
%y'&=cy-dxy.
%\end{align*}
%\end{explanation}

\begin{problem}
A more realistic assumption in the predator-prey model is that the prey population grows logistically in the absence of predators. Write the system of ODEs for this situation, using $a$ for the intrinsic growth rate and $K$ for the carrying capacity of the rabbit population:
\begin{question}

$R'= \answer{ax(1-x/K)-bxy}$
\end{question}
\begin{question}

$F'= \answer{-cx+dxy}$
\end{question}
\end{problem}

\begin{problem} Another way in which system appear is in modeling higher-order differential equations. Recall the general equation for the unforced harmonic oscillator:
\[
mx''+cx'+kx=0,
\]
where $m>0$, $c\ge 0$ and $k>0$. The variable $x$ represents the displacement of the mass. Let $y$ represent the velocity of the mass. Write below a system of ODEs for the variables $x$, $y$:

\begin{question}
$x'= \answer{y}$
\end{question}

\begin{question}
$y'= \answer{-(k/m)x-(c/m)y}$
\end{question}

%\begin{explanation} The interpretation of derivative of displacement as velocity gives directly:
%\[
%x'=y.
%\]
%Plugging this into the equation of the harmonic oscillator yields:
%\[
%y'=x''=-\frac{k}{m}x-\frac{c}{m}x'=-\frac{k}{m}x-\frac{c}{m}y.
%\]
%\end{explanation}
\end{problem}

We now proceed to the numerical and graphical interpretation of a system of ODEs. Let's go back to the predator-prey model, this time assuming concrete values for the parameters $a$, $b$, $c$ and $d$:
\begin{align}
\label{rabbit-eq-c}R'&=1.2R-0.5RF\\
\label{fox-eq-c}F'&=-F+1.5RF
\end{align}
Since we have a concrete example now, it is advisable to set specific units for all variables. Let's assume that:
\begin{align*}
t&=\text{Time, measured in weeks;}\\
R&=\text{Rabbit population, measured in thousands of individuals;}\\
F&=\text{Fox population, measured in hundreds of individuals.}
\end{align*}
Notice that we use different units for the rabbit and fox populations, since the number of predators is usually much smaller that the number of prey in a realistic situation.

Next suppose that, at a certain time, we have the values:
\[
R=4,\quad F=3
\]
With this information, we can compute the rates at which each of the populations are growing, by plugging in the values of $R$ and $F$ into the equations~\ref{rabbit-eq-c} and~\ref{fox-eq-c}:
\begin{align*}
R'&=1.2\cdot 4 - 0.5\cdot 4\cdot 3=-1.2\\
F'&=-4 + 1.5\cdot 4\cdot 3=15.0
\end{align*}
In particular, we can conclude that, for these particular sizes of the rabbit and fox population, the number of rabbits is decreasing and the number of foxes is increasing.

\begin{problem} Suppose that at a certain time, the rabbit population is  $5000$ and the fox population is $100$. Find the rate of change of the populations, and determine if each population is increasing and decreasing.

\emph{Hint}: Notice that you will have to scale the populations in the appropriate way!

\begin{question}
$R'= \answer{3.5}$

The rabbit population is 
\begin{multipleChoice}
\choice[correct]{Increasing}
\choice{Decreasing}
\end{multipleChoice}
\end{question}
\begin{question}
$F'= \answer{6.5}$

The fox population is 
\begin{multipleChoice}
\choice[correct]{Increasing}
\choice{Decreasing}
\end{multipleChoice}
\end{question}
\end{problem}

We can also represent the rates of change geometrically. We interpret the pair of derivatives $(R',F')$ as the components of a vector:
\[
\begin{bmatrix}R'\\F'\end{bmatrix}=
\begin{bmatrix}1.2R-0.5RF\\-F+1.5RF\end{bmatrix}
\]
Then, for $R=4$ and $F=3$ we have:
\[
\begin{bmatrix}R'\\F'\end{bmatrix}=
\begin{bmatrix}-1.2\\15.0\end{bmatrix}
\]
This can be represented graphically as in the figure below:
\begin{center}
\begin{tikzpicture}
	\begin{axis} [
		xlabel={$R$},
		ylabel={$F$},
		xmin=0, xmax=5,
		ymin=0, ymax=20,
		xtick={0, 1, 2, 3, 4, 5},
		ytick={0, 2, 4, 6, 8, 10, 12, 14, 16, 18, 20},
		xmajorgrids=true,
		ymajorgrids=true,
		grid style=dashed,
	]
	\end{axis}
	\draw (0,0) -- (1,1);
\end{tikzpicture}
\end{center}

\end{document}

